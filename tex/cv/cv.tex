\documentclass[margin, line]{res}  
\usepackage{helvet}
\usepackage{hyperref}
\usepackage[date=year, maxnames=3, minnames=3, style=authoryear, bibstyle=musuos]{biblatex}
\addbibresource{cv.bib}

\textheight=700pt
\newsectionwidth{1.5in}

\begin{document}
\name{Daniel de Souza Severo}
\address{\textbf{For more information}: \url{https://dsevero.com}}
\begin{resume}

\section{RESEARCH INTERESTS}
Minimum Description Length (MDL) Principle and its connections to machine learning and data compression. Currently working on lossless compression through bits-back coding and latent deep variable models with my advisors Ashish Khisti (UToronto) and Alireza Makhzani (Vector Institute).

\section{EDUCATION}
\textbf{University of Toronto}\\
{\sl Electrical \& Computer Engineering}\\
Master of Applied Science (M.A.Sc.)\hfill Started Fall 2020\\
Undergraduate Exchange Program (1 year)\hfill 2013 - 2014

\textbf{Federal University of Santa Catarina}, Brazil\hfill 2010 - 2015 \\
{\sl Bachelor of Science in Electronics Engineering}\\
First Class Honours, 99th percentile.

\section{AWARDS}
\textbf{Vector Scholarship in Artificial Intelligence Recipient 2020-21} \hfill 2020\\
The Vector Scholarship in AI supports the recruitment of top students to AI-related master’s programs in Ontario and is valued at \$17,500.\\
\url{https://vectorinstitute.ai/aimasters}

\textbf{NSERC Applied Research Rapid Response to COVID-19 Grant}\hfill 2020\\
Our project titled "Canadian Hospital Simulator For Management of COVID19 Cases and Contact Tracing" was awarded \$75,000.00.\\
\url{https://www.nserc-crsng.gc.ca/Innovate-Innover/CCI-COVID_eng.asp}

\textbf{Virtual Design Challenge Winner} \hfill 2019\\
Won 1st place at the VDC hosted by The University of British Columbia with my paper \emph{Proof of Novelty}. Received a cash prize of \$3,000.\\
\url{https://github.com/dsevero/Proof-of-Novelty}

\textbf{Student Merit Award and Medal} \hfill 2015\\
Graduated with the highest GPA ever obtained (at the time) for my major. Elected "Best Student" by the faculty of Electrical \& Electronics Engineering at the Federal University of Santa Catarina.

\textbf{Science Without Borders Scholarship} \hfill 2013\\
Awarded a full scholarship that covered tuition, transportation, necessary materials and living costs to study 2 academic semesters at the University of Toronto.

\section{PUBLICATIONS}
\nocite{*}
\printbibliography[heading=none, keyword=publication]

\section{PREPRINTS}
\printbibliography[heading=none, keyword=preprint]

\newpage
\section{TEACHING EXPERIENCE}
\textbf{Federal University of Santa Catarina}\\
{\sl Teaching Assistant}\\
Assisted professors by ministering tutorials, preparing lecture materials and helped students individually at regular office hours.

\begin{itemize}
    \item \textbf{Communications Theory} \hfill Fall and Winter 2015\\
Amplitude and frequency modulations; multiplexing; noise in communication systems; pulse modulation; analog-to-digital conversion; digital transmission in baseband and passband.
    \item \textbf{Introduction to Electronics} \hfill Fall and Winter 2013\\
Operational amplifiers; diodes; the bipolar junction transistor; field effect transistors; optoelectronic components.
    \item \textbf{Single-Variable Calculus} \hfill Fall 2010\\
Real-valued functions; limits; continuity; derivatives and applications; definite and indefinite integrals; integration techniques; improper integrals.
\end{itemize}

\textbf{CERTI Foundation} \hfill 2010 - 2013\\
{\sl Intern Programming Instructor}\\
Responsible for the technical training of new and current interns. Created a training course in LabVIEW programming that is still in use as of 2020.

\section{PROFESSIONAL SERVICE}
\textbf{NeurIPS 2019: Conference on Neural Information Processing Systems}\\
Reviewer for the Machine Learning for Health (ML4H) workshop.

\section{OPEN SOURCE CONTRIBUTIONS}
\textbf{Dask: Scalable analytics in Python}\\
\url{https://github.com/dask/dask/pulls?q=author:dsevero}

\textbf{Dask-ML: Scalable Machine Learn with Dask}\\
\url{https://github.com/dask/dask-ml/pulls?q=author:dsevero}

\textbf{Ward2ICU: A Vital Signs Dataset of Inpatients from the General Ward}\\
\url{https://github.com/3778/Ward2ICU}

\section{PROFESSIONAL EXPERIENCE}
\textbf{Vector Institute for Artificial Intelligence} \hfill 2020 - Current\\
{\sl Graduate Student Researcher}\\
\begin{small}
    Currently working on machine learning and information theory (source coding).
\end{small}

\textbf{Independent Contractor} \hfill 2018 - Current\\
{\sl Machine Learning Engineer \& Researcher}\\
\begin{small}
    Developed a Fast Healthcare Interoperability Resources DataLake for running high volume machine learning models; Feature engineering and mathematical modeling for clustering algorithms used to segment patients into similar health groups; Ranked patients by future spendings using financial data achieving a precision at n=1,000 of 50\% from a 15,000 total; Predicted patient LoS (Length of Stay) with regression techniques and hospital sensor data; Modified CoSimRank to create a similarity measure between developers and companies using Stack OverFlow data using Neo4j and Python.
\end{small}

\textbf{Linx Impulse} \hfill 2016 - 2018\\
{\sl Head of Data Science}\\
\begin{small}
    Developed recommendation algorithms for E-commerce customers; Provided ad-hoc big data analyses to find insights from our data; Designed and monitored competitive A/B experiments devised to validate our systems performance in the face of competition; Internal A/B testing tool using the SciPy and Jupyter stack; Bandit algorithms for online optimization
\end{small}

\textbf{Wavetech Technology Solutions} \hfill 2015\\
{\sl Embedded Systems Engineering Intern.}\\
\begin{small}
    Worked on microcontroller programming in C/C++ for cochlear implants.
\end{small}

\newpage
\textbf{CERTI Foundation} \hfill 2010 - 2013 (Intern.)\\
\begin{small}
    Implemented signal processing routines (filter design 
\end{small}
\hfill 2015 - 2016 (R. Eng)\\ 
\begin{small}
    and realization) in C; Programmed back-end and front-end Python software for Raspberry Pi; Embedded eLua on a platform previously developed by CERTI.
\end{small}

\textbf{WEG Industries}  \hfill Summers 2011 and 2012\\
{\sl Electrical Engineering Intern.}\\
\begin{small}
    Software upgrade, in LabVIEW, of an automatic calibrator of multimeters in order to account for different input frequencies; Conception and implementation of a hardware and software (LabVIEW) system that acquires, processes and stores data of specific parameters of electric motors.
\end{small}

\section{REFERENCES}
\textbf{Prof. Ashish Khisti} \hfill University of Toronto\\
{\sl Professor and Canada Research Chair (Tier II)}\\
{\sl Department of Electrical \& Computer Engineering}\\
\url{https://www.comm.utoronto.ca/~akhisti/}

\textbf{Prof. Alireza Makhzani} \hfill Vector Institute\\
{\sl Faculty member at the Vector Institute for Artificial Intelligence}\\
{\sl Adjunct Professor and Canada CIFAR AI Chair}\\
{\sl Department of Electrical \& Computer Engineering}\\
\url{http://www.alireza.ai/}

\textbf{Prof. Danilo Silva} \hfill Federal University of Santa Catarina\\
{\sl Associate Professor}\\
{\sl Department of Electrical and Electronic Engineering}\\
\url{http://danilosilva.sites.ufsc.br/index.html}

\textbf{Prof. Chen Feng} \hfill The University of British Columbia\\
{\sl Assistant Professor}\\
{\sl School of Engineering}\\
\url{https://people.ok.ubc.ca/cfeng01/index.html}

\textbf{Prof. Frank R. Kschischang} \hfill University of Toronto\\
{\sl Distinguished Professor of Digital Communication}\\
{\sl Department of Electrical \& Computer Engineering}\\
\url{https://www.comm.utoronto.ca/frank/}



\begin{format}
\title{l}\\
\dates{l}\location{r}\\
\body\\
\end{format}

\end{resume}
\(\)\end{document}
